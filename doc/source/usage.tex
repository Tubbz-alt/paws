\section{Usage instructions}
\label{sec:usage}

How to use \verb|paws|.

\subsection{Operations}

The library of operations built into \verb|paws|
was designed with specific workflows in mind,
and in some cases an effort was made to provide more general functionality.
Some operations have been used a lot and are known to be stable,
while others may introduce bugs if used in ways other than intended.
In order to keep the core requirements of \verb|paws| to a minimum,
\verb|paws| defaults to a state where all operations are unavailable to the user.
The user must choose which operations will be used for their workflow,
and the selected operations will attempt to import as the user enables them.
The list of enabled operations is saved in a configuration file
so that they do not need to be enabled every time \verb|paws| is run.

This section gives an overview of
the \verb|paws| operation manager, 
including how to enable and disable operations
and best practices for developing new operations.

\subsubsection{Enabling and Disabling Operations}

\subsubsection{Operation Development}


\subsection{Workflow}

\subsubsection{Building a Workflow}

\subsubsection{Execution}

\subsubsection{Batch and Realtime Execution}


\subsection{API}

\subsubsection{Python API}

\subsubsection{Command line API}



